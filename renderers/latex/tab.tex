
\documentclass{article}
\usepackage{booktabs}
\usepackage{multirow}
\usepackage[margin=1in]{geometry}
\usepackage{array}
\usepackage{crimson}
\newcolumntype{L}[1]{>{\raggedright\arraybackslash}p{#1}}
\begin{document}
%date-generated2019-11-04 16:17:01
2019-11-04 16:17:01

\begin{table}[htbp]
\centering
\setlength{\tabcolsep}{0.5em}
\def\arraystretch{1.5}  
\caption{Definition of Autoimmune diseases}
\begin{tabular}{l*{3}{c}}
\toprule
\multicolumn{4}{c}{>= 1 admission (ICD-10 codes as below) OR >= 2 physician claims (ICD-9 codes as below) }\\
\multirow{2}{*}{Disease                     }&\multicolumn{2}{c}{             ICD9 codes             }&     ICD10 codes     \\
                            & Physician claims &  Hospital data  &                     \\
\midrule
Pernicious anemia           &       281        &      281.0      &        D51.0        \\
Autoimmune hemolytic anemia &       283        &      283.0      &        D59.1        \\
Ankylosing spondylitis      &       720        &      720.0      &         M45         \\

\bottomrule
\end{tabular}
\label{table:mr}
\end{table}

\end{document}